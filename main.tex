\documentclass[10pt,a4paper,twoside]{article}

\usepackage[T1]{fontenc}
\usepackage[latin1,utf8]{inputenc}

\usepackage{lmodern}

\usepackage[pdftex]{graphicx} 

\usepackage[tracking=true]{microtype}
               
                               
\usepackage{amsmath,amssymb,amsthm}
\usepackage{xfrac}
\usepackage{mathrsfs}
\usepackage{mathtools}
\usepackage{grffile}  

\usepackage{bbm} %for use of identity-matrix
\usepackage{dsfont}

\usepackage[]{subfigure}
\usepackage{wrapfig}
\usepackage{multicol}

\usepackage{verbatim}
\usepackage{setspace}

\usepackage{color}
\usepackage[]{hyperref}
\usepackage{listings}

\usepackage{accents}
\usepackage{textcomp}
\usepackage{multirow}
\usepackage{booktabs}
\usepackage{float}

\usepackage[numberedbib]{apacite}
\bibliographystyle{apacite}
\usepackage[flushleft]{threeparttable}
\usepackage{tabulary}
\usepackage{indentfirst}


\usepackage{geometry}
\geometry{a4paper,left=30mm,right=25mm, top=20mm, bottom=25mm}

\usepackage{fancyhdr}
\pagestyle{fancy}
\fancyhf{}
\rhead{André Crescenzo}
\lhead{\small{Computer-aided Design of Bio-inspired Nanoporous Silica Materials}}
\lfoot{\today}
\rfoot{\thepage}

\usepackage{abstract}
\usepackage{authblk}



\title{Computer-aided Design of Bio-inspired Nanoporous Silica Materials}
\author[1,2]{André Crescenzo\thanks{ Corresponding author.\\ Email: \ \texttt{andre.crescenzo.2014@uni.strath.ac.uk}}}
\author[1]{Alessia Centi\thanks{ Email: \ \texttt{alessia.centi@strath.ac.uk}}}
\author[1]{Miguel Jorge\thanks{Email: \ \texttt{miguel.jorge@strath.ac.uk}}}
\affil[1]{Department of Chemical and Process Engineering, University of Strathclyde}
\affil[2]{Departamento de Engenharia Química da Escola Politécnica, Universidade de São Paulo}
\renewcommand\Authands{ and }
\date{\today \\
\begin{abstract}
\textbf{Aim:} Finish my job!\\
\textbf{Conclusion:}Repeating the results is not drawing a conclusion.
\begin{tabular}
& \textbf{Keywords}: Latex$\cdot$ Bibtex  $\cdot$ Scientific Paper $\cdot$ More Scientific Papers $\cdot$ More Scientific Papers  $\cdot$ More Scientific Papers $\cdot$ More Scientific Papers $\cdot$ More Scientific Papers $\cdot$ More Scientific Papers 
\end{tabular}
\end{abstract}}

\pagenumbering{roman}

\begin{document}
%\doublespace
\begin{titlepage}

\newcommand{\HRule}{\rule{\linewidth}{0.5mm}} 

\center 
 
\begin{figure*}[ht!]
	\includegraphics[width=1 \textwidth]{./images/StrathLogo}
\end{figure*}


\textsc{\LARGE University of Strathclyde}\\[1.5cm] 
\textsc{\Large Department of Chemical \& Process Engineering}\\[0.5cm] 
\textsc{\large M.Eng Chemical \& Process Engineering 18530}\\[0.5cm] 


\HRule \\[0.4cm]
{ \huge \bfseries Computer-aided Design of Bio-inspired Nanoporous Silica Materials}\\[0.4cm] % Title of your document
\HRule \\[1.5cm]
 

\begin{minipage}{0.4\textwidth}
\begin{flushleft} \large
\emph{Author:}\\
André \textsc{Crescenzo} 
\end{flushleft}
\end{minipage}
~
\begin{minipage}{0.4\textwidth}
\begin{flushright} \large
\emph{Supervisors:} \\
Miguel \textsc{Jorge} \\ 
Alessia \textsc{Centi} \\
Carlos F. \textsc{Rangel}
\end{flushright}
\end{minipage}\\[4cm]

{\large \today}\\[3cm] % Date, change the \today to a set date if you want to be precise


\vfill 

\end{titlepage}
\addtocontents{toc}{~\hfill\textbf{Page}\par}
\section{Summary}
\setcounter{page}{1}
...
\vfill
\newpage

\setcounter{tocdepth}{3}
\tableofcontents



\vfill
\newpage

\section{Acknowledgements}

\textit{...}

\vfill
\newpage

\pagenumbering{arabic}
\section{Introduction}
%%%%%%%%%%%%%%%%%%%%%%%%%%%%%%%%%%%%%%%%%%%%%%%%%%%%%%%%%%%%%%%%%%%%%%%%%%%%%%%%%%%%%%%%%%%%%%%%%%%%%%
%A good introduction is a clear statement of the problem or project and the reasons for studying it. 
%This information should be contained in the first few sentences.										
%Give a concise and appropriate background discussion of the problem									
%and the significance, scope, and limits of the work. Outline what has been done						
%before by citing truly pertinent literature, but do not include a general survey of					
%semirelevant literature. State how your work differs from or is related to work						
%previously published. Demonstrate the continuity from the previous work to yours.					
%The introduction can be one or two paragraphs long. Often, the heading								
%“Introduction” is not used because it is superfluous; opening paragraphs are usually introductory	
%%%%%%%%%%%%%%%%%%%%%%%%%%%%%%%%%%%%%%%%%%%%%%%%%%%%%%%%%%%%%%%%%%%%%%%%%%%%%%%%%%%%%%%%%%%%%%%%%%%%%%
%clear statement of the problem + reasons for studying it\\
%  background discussion\\
%  what has been done by others(cite works)\\
%  how mine differs from the others\\
%  how it relates to the others\\
%  significance of the work\\
%  scope of the work\\
%  limts of the work\\
%  continuity?(maybe silica interactions...)\\

"Hello World" \cite{someone}
\section{Theoretical Basis}
\section{Methods Description} 
\section{Results and Discussion}
\section{Conclusion} 
\section{Nomenclature} 
   \begin{tabulary}{1.0\textwidth}{LCL}
   $k_B$ & & Boltzmann's Constant\\
   $C_n$ &   & Molar concentration of species n ($mM$) \\
   \end{tabulary}

\bibliography{./biblio/biblio}
\vfill
\newpage
\section{Appendix}
\label{sec:appendix}
\setcounter{page}{1}
%\begin{equation}
%E_c=\frac{1}{2}mv^2
%\label{eqn:bua}
%\end{equation}
%
%\begin{table*}[ht!] 
%  \centering
%\begin{threeparttable}
%
%  \caption{Table generated by Excel2LaTeX from sheet 'Sheet1'}
%
%    \begin{tabular}{rrrrrr}
%    \toprule
%    Model & Depth & Width & Angle & Factor 1 & Factor 2 \\
%	& (mm) & (mm) & (°)  &  &  \\
%    \midrule
%    111   & 1,00  & 1,50  & 10\tnote{a}   & 357.921 & 532.289 \\
%    112   & 1,00  & 1,50  & 15   & 382.379 & 567.234 \\
%    113   & 1,00  & 1,50  & 20   & 383.863 & 569.600 \\
%    121   & 1,00  & 2,00  & 10   & 398.199 & 590.473 \\
%    122   & 1,00  & 2,00  & 15   & 486.306 & 710.483 \\
%    123   & 1,00  & 2,00  & 20   & 430.330 & 636.471 \\
%    131   & 1,00  & 2,50  & 10   & 441.735 & 654.499 \\
%    132   & 1,00  & 2,50  & 15   & 460.925 & 681.645 \\
%    133   & 1,00  & 2,50  & 20   & 469.115 & 693.700 \\
%    211   & 1,25  & 1,50  & 10   & 374.784 & 557.029 \\
%    212   & 1,25  & 1,50  & 15   & 399.053\tnote{b} & 591.402 \\
%    213   & 1,25  & 1,50  & 20   & 411.377 & 609.042 \\
%    221   & 1,25  & 2,00  & 10   & 415.050 & 615.336 \\
%    222   & 1,25  & 2,00  & 15   & 430.991 & 638.237 \\
%    223   & 1,25  & 2,00  & 20   & 455.857 & 673.613 \\
%    231   & 1,25  & 2,50  & 10   & 472.885 & 698.958 \\
%    232   & 1,25  & 2,50  & 15   & 484.567 & 715.341 \\
%    233   & 1,25  & 2,50  & 20  & 497.320 & 733.923 \\
%    311   & 1,50  & 1,50  & 10   & 385.110 & 572.665 \\
%    312   & 1,50  & 1,50  & 15   & 406.144 & 602.130 \\
%    313   & 1,50  & 1,50  & 20   & 480.613 & 706.960 \\
%    321   & 1,50  & 2,00  & 10   & 490.910 & 722.194 \\
%    322   & 1,50  & 2,00  & 15   & 513.846 & 754.804 \\
%    323   & 1,50  & 2,00  & 20   & 529.291 & 777.175 \\
%    331   & 1,50  & 2,50  & 10   & 542.184 & 796.618 \\
%    332   & 1,50  & 2,50  & 15   & 510.958 & 753.298 \\
%    333   & 1,50  & 2,50  & 20   & 527.253 & 776.981 \\
%         \midrule
%          &       &       & Sum: & 12.138.966 & 17.932.100 \\
%    \bottomrule
%    \end{tabular}%
%    \begin{tablenotes}
%    	\item[a] This should be cited with a \citeA{magic}
%    	\item[b] And this is another note
%    \end{tablenotes}
%  \label{tab:addlabel}%
%\end{threeparttable} 
%\end{table*}
%
%\begin{figure}[ht!]
%  \begin{center}
%	\includegraphics[width=0.3 \textwidth]{./images/wip}
%	\caption{Work in Progress}
%	\label{Fig:Boxplot}
%  \end{center}
%\end{figure}
%
%\textbf{Model 1 description files}
%
%\textbf{CGTraj input:}
%\begin{lstlisting}[frame=single]
% &TRAJ
%\end{lstlisting}
\end{document}