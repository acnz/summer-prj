\documentclass[10pt,a4paper,twoside]{article}

\usepackage[T1]{fontenc}
\usepackage[latin1,utf8]{inputenc}

\usepackage{lmodern}

\usepackage[pdftex]{graphicx} 

\usepackage[tracking=true]{microtype}
               
                               
\usepackage{amsmath,amssymb,amsthm}
\usepackage{xfrac}
\usepackage{mathrsfs}
\usepackage{mathtools}
\usepackage{grffile}  

\usepackage{bbm} %for use of identity-matrix
\usepackage{dsfont}

\usepackage[]{subfigure}
\usepackage{wrapfig}
\usepackage{multicol}

\usepackage{verbatim}
\usepackage{setspace}

\usepackage{color}
\usepackage[]{hyperref}
\usepackage{listings}

\usepackage{accents}
\usepackage{textcomp}
\usepackage{multirow}
\usepackage{booktabs}
\usepackage{float}

\usepackage[numberedbib]{apacite}
\bibliographystyle{apacite}
\usepackage[flushleft]{threeparttable}
\usepackage{tabulary}
\usepackage{indentfirst}


\usepackage{geometry}
\geometry{a4paper,left=30mm,right=25mm, top=20mm, bottom=25mm}

\usepackage{fancyhdr}
\pagestyle{fancy}
\fancyhf{}
\rhead{André Crescenzo}
\lhead{\small{Computer-aided Design of Bio-inspired Nanoporous Silica Materials}}
\lfoot{\today}
\rfoot{\thepage}

\usepackage{abstract}
\usepackage{authblk}



\title{Computer-aided Design of Bio-inspired Nanoporous Silica Materials}
\author[1,2]{André Crescenzo\thanks{ Corresponding author.\\ Email: \ \texttt{andre.crescenzo.2014@uni.strath.ac.uk}}}
\author[1]{Alessia Centi\thanks{ Email: \ \texttt{alessia.centi@strath.ac.uk}}}
\author[1]{Miguel Jorge\thanks{Email: \ \texttt{miguel.jorge@strath.ac.uk}}}
\affil[1]{Department of Chemical and Process Engineering, University of Strathclyde}
\affil[2]{Departamento de Engenharia Química da Escola Politécnica, Universidade de São Paulo}
\renewcommand\Authands{ and }
\date{\today \\
\begin{abstract}
\textbf{Aim:} Finish my job!\\
\textbf{Conclusion:}Repeating the results is not drawing a conclusion.
\begin{tabular}
& \textbf{Keywords}: Latex$\cdot$ Bibtex  $\cdot$ Scientific Paper $\cdot$ More Scientific Papers $\cdot$ More Scientific Papers  $\cdot$ More Scientific Papers $\cdot$ More Scientific Papers $\cdot$ More Scientific Papers $\cdot$ More Scientific Papers 
\end{tabular}
\end{abstract}}

\pagenumbering{roman}

\begin{document}
%\doublespace
\begin{titlepage}

\newcommand{\HRule}{\rule{\linewidth}{0.5mm}} 

\center 
 
\begin{figure*}[ht!]
	\includegraphics[width=1 \textwidth]{./images/StrathLogo}
\end{figure*}


\textsc{\LARGE University of Strathclyde}\\[1.5cm] 
\textsc{\Large Department of Chemical \& Process Engineering}\\[0.5cm] 
\textsc{\large M.Eng Chemical \& Process Engineering 18530}\\[0.5cm] 


\HRule \\[0.4cm]
{ \huge \bfseries Computer-aided Design of Bio-inspired Nanoporous Silica Materials}\\[0.4cm] % Title of your document
\HRule \\[1.5cm]
 

\begin{minipage}{0.4\textwidth}
\begin{flushleft} \large
\emph{Author:}\\
André \textsc{Crescenzo} 
\end{flushleft}
\end{minipage}
~
\begin{minipage}{0.4\textwidth}
\begin{flushright} \large
\emph{Supervisors:} \\
Miguel \textsc{Jorge} \\ 
Alessia \textsc{Centi} \\
Carlos F. \textsc{Rangel}
\end{flushright}
\end{minipage}\\[4cm]

{\large \today}\\[3cm] % Date, change the \today to a set date if you want to be precise


\vfill 

\end{titlepage}
\addtocontents{toc}{~\hfill\textbf{Page}\par}
\section{Summary}
\setcounter{page}{1}
...
\vfill
\newpage

\setcounter{tocdepth}{3}
\tableofcontents



\vfill
\newpage

\section{Acknowledgements}

\textit{...}

\vfill
\newpage

\pagenumbering{arabic}
\section{Introduction}
%%%%%%%%%%%%%%%%%%%%%%%%%%%%%%%%%%%%%%%%%%%%%%%%%%%%%%%%%%%%%%%%%%%%%%%%%%%%%%%%%%%%%%%%%%%%%%%%%%%%%%
%A good introduction is a clear statement of the problem or project and the reasons for studying it. 
%This information should be contained in the first few sentences.										
%Give a concise and appropriate background discussion of the problem									
%and the significance, scope, and limits of the work. Outline what has been done						
%before by citing truly pertinent literature, but do not include a general survey of					
%semirelevant literature. State how your work differs from or is related to work						
%previously published. Demonstrate the continuity from the previous work to yours.					
%The introduction can be one or two paragraphs long. Often, the heading								
%“Introduction” is not used because it is superfluous; opening paragraphs are usually introductory	
%%%%%%%%%%%%%%%%%%%%%%%%%%%%%%%%%%%%%%%%%%%%%%%%%%%%%%%%%%%%%%%%%%%%%%%%%%%%%%%%%%%%%%%%%%%%%%%%%%%%%%
%clear statement of the problem + reasons for studying it\\
%  background discussion\\
%  what has been done by others(cite works)\\
%  how mine differs from the others\\
%  how it relates to the others\\
%  significance of the work\\
%  scope of the work\\
%  limts of the work\\
%  continuity?(maybe silica interactions...)\\

Molecular Dynamics (MD) and Monte Carlo (MC) are powerful tools to simulate molecular interactions of surfactants in solvent systems, allowing a deeper understanding of their self-assembly process \cite{mjsilica}. This process results in several types of surfactant mesoscale conformations that are specially useful to design bio-inspired silica materials \cite{bioinsp}. With the addition of silica, these structures behave as scaffolds to mesoporous or nanoporous structures that are maintained even after surfactant removal, silica oligomer polymerization and calcination \cite{silica1}.
A vast range of silica materials are examples of this phenomena, such as MCM-41 as reported by \citeA{mcm}, SBA-15 \cite{sba}, MSU-V \cite{msuv} and many others, in which the self-assembled structure depends on the type of surfactant, concentration of the substances involved and synthesis conditions such as temperature and pH. It should be noted that most of the experimental methods used to obtain data are based on observation and interpretation of final silica structure by using X-ray diffraction (XRD) and transmission electron microscopy (TEM). It follows that initial self-assembled conformations are predicted as a reflex of the final results and little is known about the mechanistic of this process. However, with MD simulations it is possible to observe and analyse these initial steps of self-assembly and predict, with more accuracy, properties and frameworks provided by surfactants \cite{lipid}.

%  background discussion\\
%  what has been done by others(cite works)\\
On the other hand, a major concern is that even though MD uses sophisticated software prepared to simulate systems with thousands of atoms, using all capacities of hardware available, such as high-speed multi-core processors in conjunction with GPUs designed specifically to process data from arrays \cite{gromacs}, they can hardly achieve long time horizons and are commonly limited to a few microseconds depending on the size of the system. For this reason, several techniques have been developed to optimize the performance of the simulations such as coarse-grain methods \cite{mjsilica}. The basic idea of this technique is to fit parameters of atom groups with similar properties in a bigger "bead", which includes atomic masses and electrostatic charges lumped in approximated values. For example, given a simulation of an arbitrary surfactant with a long hydrophobic tail, it is possible to merge three or four carbon atoms (and its hydrogen atoms) in groups since they have similar hydrophobic properties, by this means reducing the number of particles in the system and speeding up the simulation. 

%  how mine differs from the others\\
%  how it relates to the others\\
In order to provide molecular topologies to simulations software packages several force-fields, for example MARTINI \cite{martini}, uses Lennard-Jones potentials fitted to a range of pre-defined bead types to describe coarse-grain models. Furthermore, not only intramolecular beads are possible, but also intermolecular beads can be specified, such as multiple solvent molecules merged in a single bead or ions surround by water molecules. Previous works conducted by \cite{mjsilica} with this method recreated a model of surfactant in the presence of silica with explicit water that was successful in describing rod-like self-assembly structures detected on MCM-41 materials, demonstrating the capacities of up-scaling this type of systems. Nevertheless, solvent presence demands most of the computational resources, hence  implicit solvent scheme has been the focus of many studies \cite{gromacs}.

Different concepts have been applied to develop a suitable model for implicit solvents; For example, \citeA{drymartini} developed the Dry MARTINI force-field by modifying parameters of its predecessor, in such manner that solvent interactions became incorporated in these values and then solvent beads are no longer necessary. Another method, described by \citeA{magic} is able to recreate an implicit solvent system from interaction potentials generated from a bottom-up approach, that means by using an all-atoms simulation to generate parameters for the coarse-grain model. It is supposed  that thermodynamic changes on the system, originated from solvent interaction with amphiphilic molecules, are incorporated in these approximated potentials. Therefore, changes in parameters as concentration may not affect coarse-grained model performance in recreating self-assembly structures \cite{dmpc}.

%  significance of the work\\
%  scope of the work\\
%  limts of the work\\
The work presented in the following experiments is an attempt to create a flexible method to upscale silica-surfactant interactions \cite{silica1} from the atomistic model to a mesoscale model with the advent of this later coarse-grain technique. The methodology applied to reach the desired model is based on the MagiC software package \cite{magic} that in conjunction with a MD simulation software, in this case GROMACS \cite{gromacs}, will provide a suitable approximation to self-assembly of amphiphilic molecules. Further explanations of the process are described in the Experimental Methods section. For the scope of this project, a bolaamphiphilic molecule called 1,12-diaminododecane (DMDD) has been chosen as surfactant because, as seen in previous research by \citeA{msuv}, it self-assembles in multilamelar vesicles that in the presence of a silica precursor are capable of generating a mesoporous structure with remarkable properties. In order to validate this structure formation and framework formation for silica oligomers, a coarse-grain approximation is a suitable option since amphiphilic molecules interaction with solvents can be described efficiently with tabulated Lennard-Jones potential interactions. As a final objective at the end of this project, an implicit water coarse-grain model for DMDD will be generated and properly validated based on MD simulations and thermodynamic properties, in order to provide a satisfactory approximation to interactions with silica oligomers in a mesoscale model. 
%  continuity?(maybe silica interactions...)\\


\section{Theoretical Basis}
\section{Methods Description} 
\section{Results and Discussion}
\section{Conclusion} 
\section{Nomenclature} 
   \begin{tabulary}{1.0\textwidth}{LCL}
   $k_B$ & & Boltzmann's Constant\\
   $C_n$ &   & Molar concentration of species n ($mM$) \\
   \end{tabulary}

\bibliography{./biblio/biblio}
\vfill
\newpage
\section{Appendix}
\label{sec:appendix}
\setcounter{page}{1}
%\begin{equation}
%E_c=\frac{1}{2}mv^2
%\label{eqn:bua}
%\end{equation}
%
%\begin{table*}[ht!] 
%  \centering
%\begin{threeparttable}
%
%  \caption{Table generated by Excel2LaTeX from sheet 'Sheet1'}
%
%    \begin{tabular}{rrrrrr}
%    \toprule
%    Model & Depth & Width & Angle & Factor 1 & Factor 2 \\
%	& (mm) & (mm) & (°)  &  &  \\
%    \midrule
%    111   & 1,00  & 1,50  & 10\tnote{a}   & 357.921 & 532.289 \\
%    112   & 1,00  & 1,50  & 15   & 382.379 & 567.234 \\
%    113   & 1,00  & 1,50  & 20   & 383.863 & 569.600 \\
%    121   & 1,00  & 2,00  & 10   & 398.199 & 590.473 \\
%    122   & 1,00  & 2,00  & 15   & 486.306 & 710.483 \\
%    123   & 1,00  & 2,00  & 20   & 430.330 & 636.471 \\
%    131   & 1,00  & 2,50  & 10   & 441.735 & 654.499 \\
%    132   & 1,00  & 2,50  & 15   & 460.925 & 681.645 \\
%    133   & 1,00  & 2,50  & 20   & 469.115 & 693.700 \\
%    211   & 1,25  & 1,50  & 10   & 374.784 & 557.029 \\
%    212   & 1,25  & 1,50  & 15   & 399.053\tnote{b} & 591.402 \\
%    213   & 1,25  & 1,50  & 20   & 411.377 & 609.042 \\
%    221   & 1,25  & 2,00  & 10   & 415.050 & 615.336 \\
%    222   & 1,25  & 2,00  & 15   & 430.991 & 638.237 \\
%    223   & 1,25  & 2,00  & 20   & 455.857 & 673.613 \\
%    231   & 1,25  & 2,50  & 10   & 472.885 & 698.958 \\
%    232   & 1,25  & 2,50  & 15   & 484.567 & 715.341 \\
%    233   & 1,25  & 2,50  & 20  & 497.320 & 733.923 \\
%    311   & 1,50  & 1,50  & 10   & 385.110 & 572.665 \\
%    312   & 1,50  & 1,50  & 15   & 406.144 & 602.130 \\
%    313   & 1,50  & 1,50  & 20   & 480.613 & 706.960 \\
%    321   & 1,50  & 2,00  & 10   & 490.910 & 722.194 \\
%    322   & 1,50  & 2,00  & 15   & 513.846 & 754.804 \\
%    323   & 1,50  & 2,00  & 20   & 529.291 & 777.175 \\
%    331   & 1,50  & 2,50  & 10   & 542.184 & 796.618 \\
%    332   & 1,50  & 2,50  & 15   & 510.958 & 753.298 \\
%    333   & 1,50  & 2,50  & 20   & 527.253 & 776.981 \\
%         \midrule
%          &       &       & Sum: & 12.138.966 & 17.932.100 \\
%    \bottomrule
%    \end{tabular}%
%    \begin{tablenotes}
%    	\item[a] This should be cited with a \citeA{magic}
%    	\item[b] And this is another note
%    \end{tablenotes}
%  \label{tab:addlabel}%
%\end{threeparttable} 
%\end{table*}
%
%\begin{figure}[ht!]
%  \begin{center}
%	\includegraphics[width=0.3 \textwidth]{./images/wip}
%	\caption{Work in Progress}
%	\label{Fig:Boxplot}
%  \end{center}
%\end{figure}
%
%\textbf{Model 1 description files}
%
%\textbf{CGTraj input:}
%\begin{lstlisting}[frame=single]
% &TRAJ
%\end{lstlisting}
\end{document}